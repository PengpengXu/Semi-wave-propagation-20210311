\subsection{Multi-time-scale}
Introduce the multi-time-scale with two-term expansion:
\begin{equation}
    T \sim T_0 + \varepsilon^2 T_1.
    \label{eq: expand time}
\end{equation}

Eq.~\eqref{eq: expand time} also expands partial differential operators about time \ensuremath{T}
\begin{equation}
    \pdv{}{T} \sim \pdv{}{T_0} + \varepsilon^2 \pdv{}{T_1}
    \label{eq: expand time dt}
\end{equation}
and
\begin{equation}
    \pdv{}{T} \sim \pdv[2]{}{T_0} + 2 \varepsilon^2 \pdv{}{T_0}{T_1} + \order{\varepsilon^4}
    \label{eq: expand time ddt}
\end{equation}

The undetermined solution of plane wave \ensuremath{W\bkt{X,T}} is expanded accordingly
\begin{equation}
    W\bkt{X,T} = W\bkt{X,T_0,T_1} \sim W_0\bkt{X,T_0,T_1} + \varepsilon^2 W_1\bkt{X,T_0,T_1} 
    \label{eq: expand solution}
\end{equation}

Note that the water potential is linear and thus needing no expansion
\begin{equation}
    \Phi\bkt{X,Z,T} = \Phi\bkt{X,Z,T_0,T_1}.
\end{equation}


\subsection{Hierarchical equations}
Substituting Eqs.~(\ref{eq: expand time}-\ref{eq: expand solution}) into Eqs.~\eqref{eq: FSI kinematic nondimension} and \eqref{eq: FSI EOM nondimension} and collecting \ensuremath{\varepsilon}, the classic perturbation procedure generates a series of hierarchical linear partial differential equations (PDE) with non-linear in-homogeneous terms on the right hind side from the second-order on. Retain up to the second-order:

\begin{itemize}
    \item \ensuremath{\order{\varepsilon^0}}:
    
    \begin{equation}
        \pdv{W_0}{T_0}
        =
        \frac{\mathrm{g}}{\tanh\bkt{H}} \pdv{\Phi}{Z} \eval_{Z=0}
        \label{eq: 1st order kinematic}
    \end{equation}
    
    \begin{equation}
    \begin{aligned}
        \beta\,{k}^{4} {\frac { \partial ^{4} W_0 }{\partial {X}^{4}}} 
        + k\tanh \left( H \right) {\frac {\partial ^{2}  W_0 }{\partial {T_0}^2}}
        + \alpha\mathrm{g}{\frac {\partial \Phi }{\partial T_0}} \eval_{Z=0}
        + \alpha\mathrm{g} W_0
        =0
    \end{aligned}
    \label{eq: 1st order combined}
    \end{equation}
    
    \begin{equation}
        \pdv[2]{\Phi}{X} + \pdv[2]{\Phi}{Z} = 0
        \label{eq: Laplace demensionless}
    \end{equation}
    
    \begin{equation}
        \pdv{\Phi}{Z} \eval_{Z=-H} = 0
        \label{eq: seabed demensionless}
    \end{equation}
    
    \item \ensuremath{\order{\varepsilon^2}}:
    
    \begin{equation}
    \begin{aligned}
        k & \tanh\bkt{H} \pdv[2]{W_1}{T_0}
        + \beta k^4 \pdv[4]{W_1}{X}
        + \alpha \mathrm{g} W_1  
        \\
        & =
        18 \frac{\beta k^4}{d^2} \Bkt{\pdv{W_0}{X}}^2 \pdv[2]{W_0}{X}
        - 2 k \tanh \bkt{H} \pdv{W_0}{T_0}{T_1}
        - \alpha \mathrm{g} \pdv{\Phi}{T_1} \eval_{Z=-H}
    \end{aligned}
    \label{eq: 2nd order combined}
    \end{equation}
\end{itemize}


\subsection{The first-order}

\subsubsection{The non-dimensional first-order solution}
First separate the potential into three parts, namely the time-dependent part \ensuremath{\varphi\bkt{T_0,t_1}}, the horizontal space-dependent part \ensuremath{\xi\bkt{X}} and the vertical space-dependent part as \ensuremath{\zeta\bkt{Z}}
\begin{equation}
    \Phi\bkt{X,Z,T_0,T_1} = \varphi\bkt{T_0,T_1} \xi\bkt{X} \zeta\bkt{Z}.
    \label{eq: phi separation}
\end{equation}

Substitution of Eq.~\eqref{eq: phi separation} into Eqs.~\eqref{eq: Laplace demensionless} and \eqref{eq: seabed demensionless} and elimination of \ensuremath{\varphi\bkt{T_0,T_1}} yeild
\begin{equation}
    \Phi\bkt{X,Z,T_0,T_1} = \varphi\bkt{T_0,T_1}  \cos \bkt{ X } \cosh\bkt{Z+H}.
    \label{eq: sol xi and zeta}
\end{equation}
Note that the integral constants of \ensuremath{\xi\bkt{X}} and \ensuremath{\zeta\bkt{Z}} merge into the undetermined time-dependent function \ensuremath{\varphi\bkt{T_0,T_1}}. We also get rid of the the arbitrary phase shift of \ensuremath{\xi\bkt{X}} because only the plane propagating wave is of interest. 

Taking the partial differentiation about \ensuremath{T_0} of Eq.~\eqref{eq: 1st order combined} and using Eqs.~\eqref{eq: 1st order kinematic} and \eqref{eq: sol xi and zeta}, we obtain
\begin{equation}
    k \tanh\bkt{H} \pdv[2]{\varphi}{T_0} + \beta k^4 \varphi + \alpha \pdv[2]{\varphi}{T_0} + \alpha \mathrm{g} \varphi = 0.
    \label{eq: ode for varphi}
\end{equation}

The solution of Eq.~\eqref{eq: ode for varphi} is
\begin{equation}
    \varphi \bkt{T_0,T_1}
    = 
    C\bkt{T_1}
    \cos \bkt{ 
    \sqrt{\frac{ \beta k^4 + \alpha \mathrm{g} }{ k \tanh\bkt{H} + \alpha }} T_0 
    + \theta\bkt{T_1}
    }.
    \label{eq: sol varphi}
\end{equation}

The coefficient in front of the regular time scale \ensuremath{T_0} in Eq.~\eqref{eq: sol varphi} is the primary angular frequency \ensuremath{\Omega_0}
\begin{equation}
    \Omega_0 = \sqrt{ \frac{ \beta k^4 + \alpha \mathrm{g} }{ k \tanh\bkt{H} + \alpha } }
    \label{eq: linear DR}
\end{equation}

Eq.~\eqref{eq: linear DR} is the non-dimensional linear dispersion relation.

Combine Eqs.~\eqref{eq: sol xi and zeta} and \eqref{eq: sol varphi} into a right-ward propagating wave as shown in Fig.~\ref{fig:sketch} 
\begin{equation}
    W_0 \left( X,T_{0},T_{1} \right) 
    = 
    A \bkt{T_1}
    \sin\bkt{ X
    - \Omega_0 T_0 
    + \theta\bkt{T_1} }
    \label{eq: sol W0}
\end{equation}
and
\begin{equation}
    \Phi\bkt{X,Z,T_0,T_1} 
    = 
    - \Omega_0
    \frac{ \cosh\bkt{Z+H} }{ \sinh\bkt{H} } 
    A\bkt{T_1} 
    \cos \bkt{ 
    X
    - \Omega_0 T_0 
    + \theta\bkt{T_1}
    }
    \label{eq: sol Phi}
\end{equation}
where \ensuremath{A\bkt{T_1}=- \frac{\sinh\bkt{H}}{\Omega_0}  C\bkt{T_1}}.

\subsubsection{The dimensional first-order solution}
Inverse utilization of Eq.~\eqref{eq: nondimensional variables} gives the dimensional first-order solution.
\begin{itemize}
    \item The dimensional first-order dispersion relation
    \begin{equation}
        \omega_0 =  \sqrt{ \frac{ k \tanh\bkt{kh} \bkt{\beta k^4 + \alpha \mathrm{g} } }{ k \tanh\bkt{kh} + \alpha } }
        \label{eq: linear dr}
    \end{equation}
    
    \item The dimensional first-order phase velocity
\begin{equation}
    c = \sqrt{ \frac{ \tanh\bkt{kh} \bkt{\beta k^4 + \alpha \mathrm{g} } }{ k \bkt{ k \tanh\bkt{kh} + \alpha } } }
\end{equation}
\end{itemize}


\subsection{The second-order}
\subsubsection{The non-dimensional second-order solution}
Substitute the first-order solution Eqs.~\eqref{eq: sol Phi} and \eqref{eq: sol W0} into the right-hand-side of Eq.~\eqref{eq: 2nd order combined}, then collect the trigonometric. There exist primary trigonometric terms, i.e., \ensuremath{\sin\bkt{ X - \Omega_0 T_0 + \theta\bkt{T_1}}} and \ensuremath{\cos\bkt{ X - \Omega_0 T_0 + \theta\bkt{T_1}}}. Their coefficients are
\begin{equation}
    \begin{aligned}
        C_{\sin}
        = & 
        {\frac {2\,k\tanh \left( H \right) \sqrt {\beta\,{k}^{4}+\alpha\mathrm{g}} }{\sqrt {k\tanh \left( H \right) +\alpha}}}  A \left( T_{1} \right)  {\frac {{\rm d} \theta \left( T_{1} \right)}{{\rm d}T_{1}}}
        + {\frac {9\,\beta\,{k}^{2} }{2\,{d}^{2}}}  A^{3}  \left( T_{1} \right)  \\
        & + \frac {\cosh \left( H \right) \alpha\mathrm{g} }{\sinh \left( H \right) }   {\frac {\sqrt {\beta\,{k}^{4}+\alpha\mathrm{g}}}{\sqrt {k\tanh \left( H \right) +\alpha}}}  A \left( T_{1} \right) {\frac {{\rm d} \theta \left( T_{1} \right)}{{\rm d}T_{1}}}
    \end{aligned}
    \label{eq: coef sin}
\end{equation}

\begin{equation}
    \begin{aligned}
        C_{\cos}
        = & 
        2\,{\frac {k\tanh \left( H \right)   \sqrt {\beta\,{k}^{4}+\alpha\mathrm{g}}}{\sqrt {k\tanh \left( H \right) +\alpha}}}   {\frac {{\rm d} A \left( T_{1} \right)}{{\rm d}T_{1}}}   \\
        & + \frac {\cosh \left( H \right) \alpha\mathrm{g} }{\sinh \left( H \right) } 
        {\frac {\sqrt {\beta\,{k}^{4}+\alpha\mathrm{g}}}{\sqrt {k\tanh \left( H \right) +\alpha}}}  \frac {{\rm d} A \left( T_{1} \right)}{{{\rm d} }T_{1}}
    \end{aligned}
    \label{eq: coef cos}
\end{equation}

Avoiding secular terms requires that the \ensuremath{C_{\sin}} and \ensuremath{C_{\cos}} equal to zero. Eq.~\eqref{eq: coef cos} can solve \ensuremath{A\bkt{T_1}}:
\begin{equation}
    A\bkt{T_1} = A.
    \label{eq: sol C4}
\end{equation}
Hence A is a arbitrary non-dimensional constant independent of the slow time-scale \ensuremath{T_1}.

Substitution of Eq.~\eqref{eq: sol C4} into Eq.~\eqref{eq: coef sin} leads to
\begin{equation}
    \theta \left( T_{1} \right) 
    = 
    -\frac {9\,\beta\,{k}^{2}  \sinh \left( 2\,H \right) }{2\,{d}^{2} \left( \alpha\mathrm{g}\cosh \left( 2\,H \right) +2\,k\cosh \left( 2\,H \right) +\alpha\mathrm{g}-2\,k \right) \Omega_0 }
    {A}^{2} T_{1}.
    \label{eq: theta2}
\end{equation}

Substitute Eqs.~\eqref{eq: sol C4} and \eqref{eq: theta2} back into the first-order solution Eq.~\eqref{eq: sol W0}, and let \ensuremath{A=1}. Now, the plane propagating wave reads

\begin{equation}
    \begin{aligned}
    W_{0}
    =
    \sin\bkt{ X
    - \Omega_0 T
    -\frac {9\varepsilon^2 \beta{k}^{2}  \sinh \left( 2H \right) T }{2{d}^{2} \left( \alpha \mathrm{g} \cosh \left( 2H \right) +2k\cosh \left( 2H \right) +\alpha \mathrm{g}-2k \right) \Omega_0 }
     } .
    \end{aligned}
    \label{eq: sol W0 2nd}
\end{equation}

The nonlinear angular frequency up to the second-order is
\begin{equation}
    \begin{aligned}
        \Omega 
        =
        \Omega_0 + \frac {9\varepsilon^2 \beta{k}^{2}  \sinh \left( 2H \right) }{2{d}^{2} \left( \alpha \mathrm{g} \cosh \left( 2H \right) +2k\cosh \left( 2H \right) +\alpha \mathrm{g}-2k \right) \Omega_0 }.
    \end{aligned}
    \label{eq: sol Omega 2nd}
\end{equation}


\subsubsection{The dimensional second-order solution}
Inverse utilization Eq.~\eqref{eq: nondimensional variables} gives the dimensional second-order solution, too.
\begin{itemize}
    \item The dimensional second-order dispersion relation
    \begin{equation}
    \begin{aligned}
        \omega 
        = & 
        \sqrt {\frac {{\beta\,{k}^{4}+\alpha\mathrm{g}}}{ {k\tanh \left( kh \right) +\alpha}}} 
        + 
        \frac{ 9 \varepsilon^2 \beta k^2 \tanh\bkt{kh} }{ 2 d^2 \bkt{ 2 k \tanh^2\bkt{kh} + \alpha \mathrm{g} } }
        \sqrt{\frac
        {k\tanh \left( kh \right) +\alpha}
        {\beta{k}^{4}+\alpha\mathrm{g}} }
    \end{aligned}
    \label{eq: 2.34}
\end{equation}
    
    \item The dimensional second-order phase velocity
    \begin{equation}
    \begin{aligned}
        c 
        = & 
        \frac{1}{k} \sqrt {\frac {{\beta\,{k}^{4}+\alpha\mathrm{g}}}{ {k\tanh \left( kh \right) +\alpha}}} 
        + 
        \frac{ 9 \varepsilon^2 \beta k \tanh\bkt{kh} }{ 2 d^2 \bkt{ 2 k \tanh^2\bkt{kh} + \alpha \mathrm{g} } }
        \sqrt{\frac
        {k\tanh \left( kh \right) +\alpha}
        {\beta{k}^{4}+\alpha\mathrm{g}} }
    \end{aligned}
\end{equation}
\end{itemize}